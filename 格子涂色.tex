\ifx\allfiles\undefined
%以下为头部区
%注意\begin{document}被包含在文件内

\documentclass[UTF8]{ctexrep}

%使用的包
\usepackage{CJK}
\usepackage{titlesec}
\usepackage{indentfirst}
\usepackage{graphicx}
\usepackage{listings} 
\usepackage{amssymb}
\usepackage{geometry}
\usepackage{xcolor}

%控制边距
\newgeometry{top=3.5cm, bottom=2.5cm}

%控制section的格式
\renewcommand\thesection {{Problem \Alph{section}.}}

%控制代码的格式
\lstset{
	keywordstyle=\color{blue!70},
	commentstyle=\color{red!50!green!50!blue!50},
	rulesepcolor=\color{red!20!green!20!blue!20},
	morekeywords={*,define,*,include...},
	stringstyle=\color{purple},
	tabsize=2,
	escapeinside=``,
} 

\begin{document}
%英文和中文的字体
\ttfamily
\songti
%去除英文和中文之间的空格
\CJKsetecglue{}
\fi
% --封印线--


\section{格子涂色}
\subsection*{描述}
木下吉子有一个$n\times n$的方格,一开始所有格子都是没有涂色的。每一轮,她会随机选择所有格子中的一个点(x,y)作为左上角,并以点($n,n$)为右下角,将这两个点所组成的矩形内的所有格子涂色。
求经过$k$轮操作后涂色格子数目的期望值。


\subsection*{输入数据}
第一行为测试数据的组数$T(T\leqslant 50)$

每组测试数据有两个整数$n$和$k(1\leqslant n\leqslant 1000,1\leqslant k\leqslant 100)$

\subsection*{输出数据}
对于每组测试数据,输出一行结果,四舍五入保留两位小数。

\subsection*{样例输入}
\noindent 2\\
5 2\\
2 1

\subsection*{样例输出}
\noindent 13.16\\
2.25


% --封印线--
\ifx\allfiles\undefined
\end{document}
\fi
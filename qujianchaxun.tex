\ifx\allfiles\undefined
\documentclass[UTF8]{ctexrep}
\usepackage{CJK}
\usepackage{indentfirst}
\usepackage{graphicx}
\usepackage{amssymb}
\usepackage{geometry}
\newgeometry{top=3.5cm, bottom=3cm}
\renewcommand\thesection
    {{\Alph{section}}}

\begin{document}
\tt
\CJKfamily{song}
\fi
% --封印线--


\section{区间查询}
\subsection*{描述}
木下吉子有一个长度为$n$的数组$A$,接着有$m$个查询。

每个查询给一个区间$[L,R](1\leqslant L\leqslant R\leqslant n)$,请你输出该区间的最大值$max$,最左端最大值的下标$l$和最右端最大值的下标$r$。

\subsection*{输入数据}
第一行为测试数据的组数$T(1\leqslant T\leqslant 10)$

每组数据第一行有一个整数$n(1\leqslant n\leqslant 10^6)$

第二行共$n$个整数,其中第$i$个整数是$A_i(-10^9\leqslant A_i\leqslant 10^9)$

第三行,一个整数$m(1\leqslant m\leqslant 10^5)$

接下去$m$行,每行两个整数$L$和$R(1\leqslant L\leqslant R\leqslant n)$

\subsection*{输出数据}
对于每组测试数据,输出$m$行,每行3个整数$max,l,r,$用空格隔开。

每组测试数据结束后再输出一个空行。

\subsection*{样例输入}
\noindent 2\\
5\\
-1 9 2 9 3\\
2\\
2 5\\
1 1\\
3\\
3 4 5\\
1\\
1 3

\subsection*{样例输出}
\noindent 9 2 4\\
-1 1 1\\

\noindent 5 3 3


% --封印线--
\ifx\allfiles\undefined
\end{document}
\fi
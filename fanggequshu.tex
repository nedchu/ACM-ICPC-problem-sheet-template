\ifx\allfiles\undefined
\documentclass[UTF8]{ctexrep}
\usepackage{CJK}
\usepackage{indentfirst}
\usepackage{graphicx}
\usepackage{amssymb}
\usepackage{geometry}
\newgeometry{top=3.5cm, bottom=3cm}
\renewcommand\thesection
    {{\Alph{section}}}

\begin{document}
\tt
\CJKfamily{song}
\fi
% --封印线--


\section{方格取数}
\subsection*{描述}
木下吉子有一个$n\times n$的网格卡片,每个方格中含有一个数字。木下吉子想到了一个游戏,他想以最小代价划掉所有网格中的数字。每次在网格中没被划掉的位置中选择一个位置,然后划掉这个位置所在的行和列的所有方格,把选定位置的数字累加到代价中。同一位置可以被划掉多次,但被划掉的位置不能被选择,开始所有数字没被划掉,初始代价为0。

\subsection*{输入数据}
第一行为数据的组数$T(T\leqslant 100)$

每组数据第一行为网格的规模$n(n\leqslant 100)$

接下来$n$行每行$n$个数字,$a_{i,j}$表示网格中$i$行$j$列的数字$(1\leqslant a_{i,j}\leqslant 10^6)$

\subsection*{输出数据}
一行数字,划掉所有数字的最小代价。

\subsection*{样例输入}
\noindent 1\\
2\\
1 2\\
2 5

\subsection*{样例输出}
\noindent 4


% --封印线--
\ifx\allfiles\undefined
\end{document}
\fi
\ifx\allfiles\undefined
\documentclass[UTF8]{ctexrep}
\usepackage{CJK}
\usepackage{titlesec}
\usepackage{indentfirst}
\usepackage{graphicx}
\usepackage{listings} 
\usepackage{amssymb}
\usepackage{geometry}
\newgeometry{top=3.5cm, bottom=2.5cm}
\renewcommand\thesection
    {{Problem \Alph{section}.}}
    

% 导言区结束

\begin{document}
% 英文和中文的字体

\tt
\CJKfamily{song}

% 用于拆分文件的宏
\def\allfiles{}

% 标题的页面,包含ICPC标志,作者,时间以及校徽
\begin{titlepage}
    \centering
    \ 
    \includegraphics[width=6cm]{{icpc.jpg}}\\
    \vspace{1.5cm}
    {\huge 第七届ECNU Coder程序设计竞赛\ 高年级组}\\   
    \vspace{1.5cm}
    {\Large 作者:储德明\ 陈锴\ 王玮鑫}\\
    \vspace{0.5cm}
    {\Large 2016年5月25日}\\
    \vspace{2cm}
    \includegraphics[width=6cm]{{ecnu.jpg}}
\end{titlepage}


% 空白页
\pagebreak
\  
\thispagestyle{empty}
\pagebreak

% 所有的试题
\input{gao}
\pagebreak
\input{gcd}
\pagebreak
\ifx\allfiles\undefined
\documentclass[UTF8]{ctexrep}
\usepackage{CJK}
\usepackage{indentfirst}
\usepackage{graphicx}
\usepackage{amssymb}
\usepackage{geometry}
\newgeometry{top=3.5cm, bottom=3cm}
\renewcommand\thesection
    {{\Alph{section}}}

\begin{document}
\tt
\CJKfamily{song}
\fi
% --封印线--


\section{数正方形}
\subsection*{描述}
木下吉子有一个$n\times n$的方阵,元素为'.'或'*'。

木下吉子想知道有多少用'*'构成的正方形。

可以使用左上角坐标$x,y$以及边长$a$来描述正方形$(a\geqslant 1)$

则有$g_{x+i,y}=g_{x,y+i}=g_{x+a-1,y+i}=g_{x+i,y+a-1}=$'*'$(0\leqslant i<a)$

\subsection*{输入数据}
第一行为测试数据组数$T(T\leqslant 30)$

每组数据第一行为$n$,即矩阵的大小$(1\leqslant n\leqslant 1000)$

接下来$n$行每行有$n$个字符,为'.'或'*'。

\subsection*{输出数据}
给定矩阵中正方形的个数。

\subsection*{样例输入}
\noindent 2
3\\
****\\
***.\\
...\\
5 \\
****** \\
****** \\
****.. \\
***... \\
******

\subsection*{样例输出}
\noindent 6\\
29


% --封印线--
\ifx\allfiles\undefined
\end{document}
\fi
\pagebreak
\ifx\allfiles\undefined
\documentclass[UTF8]{ctexrep}
\usepackage{CJK}
\usepackage{indentfirst}
\usepackage{graphicx}
\usepackage{amssymb}
\usepackage{geometry}
\newgeometry{top=3.5cm, bottom=3cm}
\renewcommand\thesection
    {{\Alph{section}}}

\begin{document}
\tt
\CJKfamily{song}
\fi
% --封印线--


\section{暗之高年级竞赛}
\subsection*{描述}
木下吉子听说高年级竞赛三人一支队伍,规则是每个队伍中至少有两人在一个寝室。为了增加气氛,木下吉子打算举办暗之高年级竞赛,其规则与高年级竞赛背道而驰,每个队伍的三个必须来自不同寝室。

有$n$个人,$m$个寝室,给定每个人的寝室号$a_i(1\leqslant a_i \leqslant m)$,问最多能够组成多少支队伍。

需要注意的是为了公平竞争,所有队伍都得由三人组成。

\subsection*{输入数据}
第一行为数据的组数$T(T\leqslant 100)$

每组数据第一行为$n,m(1\leqslant n,m\leqslant 10^3)$

接下一行有$n$个数$a_1\cdots a_n(1\leqslant a_i\leqslant m)$,$a_i$是第$i$个人的寝室编号。

\subsection*{输出数据}
一行数字,最多多少支队伍。

\subsection*{样例输入}
\noindent 2\\
3 3\\
1 2 3\\
6 4\\
1 1 2 2 3 4

\subsection*{样例输出}
\noindent 1\\
2


% --封印线--
\ifx\allfiles\undefined
\end{document}
\fi
\pagebreak
\input{xianduanshu++}
\pagebreak
\ifx\allfiles\undefined
\documentclass[UTF8]{ctexrep}
\usepackage{CJK}
\usepackage{indentfirst}
\usepackage{graphicx}
\usepackage{amssymb}
\usepackage{geometry}
\newgeometry{top=3.5cm, bottom=3cm}
\renewcommand\thesection
    {{\Alph{section}}}

\begin{document}
\tt
\CJKfamily{song}
\fi
% --封印线--


\section{软件安装}
\subsection*{描述}
木下吉子有一些软件包要安装, 这些软件之间存在依赖关系, 如果包$x$依赖包$y$, 则$y$必须在$x$之前安装。他想知道自己能否顺利将这些软件全部成功安装。

\subsection*{输入数据}
第一行为测试数据的组数$T(1\leqslant T\leqslant 20)$

每组数据中第一行为两个整数$n$和$m$, 表示软件包的数目和依赖关系数目$(1\leqslant n, m\leqslant 10^5)$

接下去$m$行, 每行两个整数$x$和$y$, 表示$x$依赖$y(1\leqslant x, y\leqslant n)$

\subsection*{输出数据}
对于每组测试数据,如果能够全部成功安装,输出YES,否则输出NO。

\subsection*{样例输入}
\noindent 2\\
3 3\\
1 2\\
2 3\\
3 1\\
3 2\\
1 2\\
2 3

\subsection*{样例输出}
\noindent NO\\
YES


% --封印线--
\ifx\allfiles\undefined
\end{document}
\fi
\pagebreak
\input{fanggequshu}
\pagebreak
\ifx\allfiles\undefined
\documentclass[UTF8]{ctexrep}
%以下为头部区
%注意\begin{document}被包含在文件内

%使用的包
\usepackage{CJK}
\usepackage{titlesec}
\usepackage{indentfirst}
\usepackage{graphicx}
\usepackage{listings} 
\usepackage{amssymb}
\usepackage{geometry}
\usepackage{xcolor}
%控制边距
\newgeometry{top=3.5cm, bottom=2.5cm}
%控制section的格式
\renewcommand\thesection {{Problem \Alph{section}.}}

%控制代码的格式
\lstset{
keywordstyle=\color{blue!70},
commentstyle=\color{red!50!green!50!blue!50},
rulesepcolor=\color{red!20!green!20!blue!20},
morekeywords={*,define,*,include...},
stringstyle=\color{purple},
tabsize=2,
escapeinside=``,
}

\begin{document}
%英文和中文的字体
\ttfamily
\CJKfamily{song}
\fi
% --封印线--


\section{集合维护}
\subsection*{描述}
木下吉子有一个集合$S$,初始有$S=\varnothing$ 。木下吉子定义了两种集合上的操作。

第一种是交操作,表示为$1\ P$,执行$S\leftarrow S\bigcap P$。

第二种是并操作,表示为$2\ P$,执行$S\leftarrow  S\bigcup P$。

$P$是一个区间。木下吉子想要知道经过$m$次操作后$S$的情况。

\subsection*{输入数据}
第一行为数据的组数$T(T\leqslant 150)$

每组数据第一行为$m$代表对$S$操作的次数$(m\leqslant 100)$

接下来$m$行按照顺序给出对集合的操作,每次操作为$1\ P$或者$2\ P$。

$P$的格式为$Ll,rR$其中$L\in\big\{(,[\big\} \qquad 
0\leqslant l,r \leqslant 100 \qquad 
R\in\big\{),]\big\}$ 

\subsection*{输出数据}
每组数据输出一行,表示集合的情况。

空集输出"empty set"。

如果集合中有多段,使用U(英文字母大写U)连接每段集合,代表并。

集合以最少集合的并集输出。

\subsection*{样例输入}
\noindent 4\\
1\\
1 (1,2)\\
2\\
2 (1,2)\\
2 (2,3)\\
2\\
2 (1,2)\\
2 [2,3)\\
3\\
2 [2,3]\\
2 [4,5]\\
1 (2,5)

\subsection*{样例输出}
\noindent empty set\\
(1,2)U(2,3)\\
(1,3)\\
(2,3]U[4,5)



% --封印线--
\ifx\allfiles\undefined
\end{document}
\fi
\pagebreak
\ifx\allfiles\undefined
\documentclass[UTF8]{ctexart}
\usepackage{CJK}
\usepackage{indentfirst}
\usepackage{graphicx}
\usepackage{amssymb}
\usepackage{geometry}
\newgeometry{top=3.5cm, bottom=3cm}
\renewcommand\thesection
    {{\Alph{section}}}

\begin{document}
\tt
\CJKfamily{song}
\fi
% --封印线--


\section{格子涂色}
\subsection*{描述}
木下吉子有一个$n\times n$的方格,一开始所有格子都是没有涂色的。每一轮,她会随机选择所有格子中的一个点(x,y)作为左上角,并以点($n,n$)为右下角,将这两个点所组成的矩形内的所有格子涂色。
求经过$k$轮操作后涂色格子数目的期望值。


\subsection*{输入数据}
第一行为测试数据的组数$T(T\leqslant 50)$

每组测试数据有两个整数$n$和$k(1\leqslant n\leqslant 1000,1\leqslant k\leqslant 100)$

\subsection*{输出数据}
对于每组测试数据,输出一行结果,四舍五入保留两位小数。

\subsection*{样例输入}
\noindent 2\\
5 2\\
2 1

\subsection*{样例输出}
\noindent 13.16\\
2.25


% --封印线--
\ifx\allfiles\undefined
\end{document}
\fi
\pagebreak
\ifx\allfiles\undefined
\documentclass[UTF8]{ctexrep}
\usepackage{CJK}
\usepackage{indentfirst}
\usepackage{graphicx}
\usepackage{amssymb}
\usepackage{geometry}
\newgeometry{top=3.5cm, bottom=3cm}
\renewcommand\thesection
    {{\Alph{section}}}

\begin{document}
\tt
\CJKfamily{song}
\fi
% --封印线--


\section{区间查询}
\subsection*{描述}
木下吉子有一个长度为$n$的数组$A$,接着有$m$个查询。

每个查询给一个区间$[L,R](1\leqslant L\leqslant R\leqslant n)$,请你输出该区间的最大值$max$,最左端最大值的下标$l$和最右端最大值的下标$r$。

\subsection*{输入数据}
第一行为测试数据的组数$T(1\leqslant T\leqslant 10)$

每组数据第一行有一个整数$n(1\leqslant n\leqslant 10^6)$

第二行共$n$个整数,其中第$i$个整数是$A_i(-10^9\leqslant A_i\leqslant 10^9)$

第三行,一个整数$m(1\leqslant m\leqslant 10^5)$

接下去$m$行,每行两个整数$L$和$R(1\leqslant L\leqslant R\leqslant n)$

\subsection*{输出数据}
对于每组测试数据,输出$m$行,每行3个整数$max,l,r,$用空格隔开。

每组测试数据结束后再输出一个空行。

\subsection*{样例输入}
\noindent 2\\
5\\
-1 9 2 9 3\\
2\\
2 5\\
1 1\\
3\\
3 4 5\\
1\\
1 3

\subsection*{样例输出}
\noindent 9 2 4\\
-1 1 1\\

\noindent 5 3 3


% --封印线--
\ifx\allfiles\undefined
\end{document}
\fi
\pagebreak
\input{xianduanshu}
\pagebreak


\end{document}
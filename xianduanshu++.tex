\ifx\allfiles\undefined
\documentclass[UTF8]{ctexrep}
\usepackage{CJK}
\usepackage{indentfirst}
\usepackage{graphicx}
\usepackage{amssymb}
\usepackage{geometry}
\usepackage{listings} 
\newgeometry{top=3.5cm, bottom=3cm}
\renewcommand\thesection
    {{\Alph{section}}}

\begin{document}
\tt
\CJKfamily{song}
\fi
% --封印线--


\section{线段树++}
\subsection*{描述}
最近木下吉子在学习线段树,她有一份线段树的代码,线段树中每个节点维护了一条L到R的线段,节点标号为rt,木下吉子想要知道调用st.build(1,1,n)之后rt的最大值。
\begin{lstlisting}[language=C++]
struct SegmentTree {
	void build(long long rt, long long L, long long R) {
		if (L != R) {
			long long mid = (L + R) / 2;
			build(rt * 2, L, mid);
			build(rt * 2 + 1, mid+1, R);
		}
	}
};

SegmentTree st;
\end{lstlisting} 

\subsection*{输入数据}
第一行为数据的组数$T(T\leqslant 200)$

每组数据一行,为$n$的大小$(1\leqslant n\leqslant 10^{18})$

\subsection*{输出数据}
一行数字,规模为$n$的线段树中rt的最大值。


\subsection*{样例输入}
\noindent 3\\
1\\
2\\
5


\subsection*{样例输出}
\noindent 1\\
3\\
9


% --封印线--
\ifx\allfiles\undefined
\end{document}
\fi
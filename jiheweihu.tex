\ifx\allfiles\undefined
\documentclass[UTF8]{ctexrep}
%以下为头部区
%注意\begin{document}被包含在文件内

\documentclass[UTF8]{ctexrep}

%使用的包
\usepackage{CJK}
\usepackage{titlesec}
\usepackage{indentfirst}
\usepackage{graphicx}
\usepackage{listings} 
\usepackage{amssymb}
\usepackage{geometry}
\usepackage{xcolor}

%控制边距
\newgeometry{top=3.5cm, bottom=2.5cm}

%控制section的格式
\renewcommand\thesection {{Problem \Alph{section}.}}

%控制代码的格式
\lstset{
	keywordstyle=\color{blue!70},
	commentstyle=\color{red!50!green!50!blue!50},
	rulesepcolor=\color{red!20!green!20!blue!20},
	morekeywords={*,define,*,include...},
	stringstyle=\color{purple},
	tabsize=2,
	escapeinside=``,
} 

\begin{document}
%英文和中文的字体
\ttfamily
\songti
%去除英文和中文之间的空格
\CJKsetecglue{}
\fi
% --封印线--


\section{集合维护}
\subsection*{描述}
木下吉子有一个集合$S$,初始有$S=\varnothing$ 。木下吉子定义了两种集合上的操作。

第一种是交操作,表示为$1\ P$,执行$S\leftarrow S\bigcap P$。

第二种是并操作,表示为$2\ P$,执行$S\leftarrow  S\bigcup P$。

$P$是一个区间。木下吉子想要知道经过$m$次操作后$S$的情况。

\subsection*{输入数据}
第一行为数据的组数$T(T\leqslant 150)$

每组数据第一行为$m$代表对$S$操作的次数$(m\leqslant 100)$

接下来$m$行按照顺序给出对集合的操作,每次操作为$1\ P$或者$2\ P$。

$P$的格式为$Ll,rR$其中$L\in\big\{(,[\big\} \qquad 
0\leqslant l,r \leqslant 100 \qquad 
R\in\big\{),]\big\}$ 

\subsection*{输出数据}
每组数据输出一行,表示集合的情况。

空集输出"empty set"。

如果集合中有多段,使用U(英文字母大写U)连接每段集合,代表并。

集合以最少集合的并集输出。

\subsection*{样例输入}
\noindent 4\\
1\\
1 (1,2)\\
2\\
2 (1,2)\\
2 (2,3)\\
2\\
2 (1,2)\\
2 [2,3)\\
3\\
2 [2,3]\\
2 [4,5]\\
1 (2,5)

\subsection*{样例输出}
\noindent empty set\\
(1,2)U(2,3)\\
(1,3)\\
(2,3]U[4,5)



% --封印线--
\ifx\allfiles\undefined
\end{document}
\fi